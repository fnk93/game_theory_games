
\documentclass{article}
\begin{document}
{{ content }}
Bei der gegebenen Spielmatrix eines 2-Personen-Nullsummenspiels\\
\begin{gather*}
\begin{pmatrix*}
{{ gamematrix }}
\end{pmatrix*}
\end{gather*}
ergeben sich in reinen Strategien folgende Kennzahlen:\\
Maximin-Strategie(n) für Spieler 1: {{ maxmin1 }}\\
Maximin-Strategie(n) für Spieler 1: {{ maxmin1 }}\\
Indeterminiertheitsintervall: {{ indet }}\\
Das Spiel ist {{ det }}\\
Aus dem unteren Spielwert von Spieler 1: {{ lowval1 }}\\
und dem unteren Spielwert von Spieler 2: {{ lowval2} }\\
ergibt sich der Garantiepunkt in reinen Strategien: {{ guar }}
Um die Bayes-Strategie zu ermitteln muss die maximale Auszahlung bei gegebener Gegnerstrategie betrachtet werden.\\
Für Spieler 1 müssen deshalb bei gegebener Strategie {{ bay1 }} von Spieler 2 die Auszahlungen {{ pay1 }} betrachtet werden.\\
Hieraus ergibt sich die Bayes-Strategie: {{ baystrat1 }}\\
Für Spieler 2 müssen deshalb bei gegebener Strategie {{ bay2 }} von Spieler 1 die Auszahlungen {{ pay2 }} betrachtet werden.\\
Hieraus ergibt sich die Bayes-Strategie: {{ baystrat2 }}\\
Das Erfüllen der Optimalitätsbedingung der Strategiekombinationen über beide Spieler aufsummiert sieht wie folgt aus:\\
\begin{gather*}
\begin{pmatrix*}
{{ ggwmatr }}
\end{pmatrix*}
\end{gather*}
Wobei sich für Spieler 1 folgende Verteilung der Erfüllung der Optimalitätsbedingung\\
\begin{gather*}
\begin{pmatrix*}
{{ ggwmatr1 }}
\end{pmatrix*}
\end{gather*}
und sich für Spieler 2 folgende Verteilung der Erfüllung der Optimalitätsbedingung ergab\\
\begin{gather*}
\begin{pmatrix*}
{{ ggwmatr2 }}
\end{pmatrix*}
\end{gather*}
{{ puresolve }}\\
{{ solvemixed }}
\end{document}

