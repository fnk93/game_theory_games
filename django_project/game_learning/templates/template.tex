
\documentclass{article}
\usepackage[ngerman]{babel}
\usepackage{fontenc,inputenc,graphicx,makeidx,fancyhdr,array,colortbl,xcolor,longtable,xspace}
\usepackage{amsmath}
\usepackage{amssymb}
\usepackage{mathtools}
\nonstopmode
\begin{document}
{{ content }}
Bei der gegebenen Spielmatrix eines Zwei-Personennullsummenspiels:\\
\begin{gather*}
\begin{pmatrix*}
& {{col}} &\\
\end{pmatrix*}
\end{gather*}
ergeben sich folgende Berechnungen:
\begin{itemize}
\item Maximin-Strategie(n) von Spieler 1: {{ dict.maximin1 }}
\item Maximin-Strategie(n) von Spieler 2: {{ dict.maximin2 }}
\item Indeterminiertheitsintervall: {{ dict.indet_intervall }}, weshalb das Spiel determiniertindeterminiert ist.
\item Unterer Spielwert Spieler 1: {{ dict.lower_value1 }}
\item Unterer Spielwert Spieler 2: {{ dict.lower_value2 }}
\item Garantiepunkt in reinen Strategien: {{ dict.guarantee_point }}
\item Bei gespielter Strategie {{ dict.baystrats1strat2 }} von Spieler 2, m\"ussen f\"ur Spieler 1 folgende Auszahlungen betrachtet werden:
\end{itemize}
\begin{center}\({{ dict.baystrats1watch1 }}\)\end{center}\\
\begin{itemize}
\item Hieraus ergeben sich die Bayes-Strategien von Spieler 1 gegen\"uber Strategie {{ dict.baystrats1strat2 }}: {{ dict.baystrats1 }}
\item Bei gespielter Strategie {{ dict.baystrats2strat1 }} von Spieler 1, m\"ussen f\"ur Spieler 2 folgende Auszahlungen betrachtet werden:
\end{itemize}
\begin{center}\({{ dict.baystrats2watch2 }}\)\end{center}\\
\begin{itemize}
\item Hieraus ergeben sich die Bayes-Strategien von Spieler 2 gegen\"uer Strategie {{ dict.baystrats2strat1 }}: {{ dict.baystrats2 }}
\end{itemize}
Die Erf\"ullung der Optimalit\"atsbedingung sieht \"uber beide Spieler wie folgt aus:
\begin{gather*}
\begin{pmatrix*}
 & {{col}} &\\
\end{pmatrix*}
\end{gather*}
Die Erf\"ullung der Optimalit\"atsbedingung sieht f\"ur Spieler 1 wie folgt aus:
\begin{gather*}
\begin{pmatrix*}
 & {{col}} &\\
\end{pmatrix*}
\end{gather*}
Die Erf\"ullung der Optimalit\"atsbedingung sieht f\"ur Spieler 2 wie folgt aus:
\begin{gather*}
\begin{pmatrix*}
 & {{col}} &\\
\end{pmatrix*}
\end{gather*}
Darum existier(en) Gleichgewicht(e) mit den Strategie-Kombinationen: {{ dict.nggw_pure }}
Da keine Strategien-Kombination, sowohl f\"ur Spieler 1, als auch f\"ur Spieler 2 optimal ist, existieren keine Nash-Gleichgewichte in reinen Strategien.
Nun werden die Gleichgewichte in gemischten Strategien gesucht:
Da es sich um ein Zwei-Personennullsummenspiel handelt, kann das Spiel sowohl mit der Simplex-Methode, als auch durch die L\"osung von linearen Gleichungssystemen gel\"ost werden.
Um den Simplex-Algorithmus bei einem Zwei-Personennullsummenspiel anwenden zu k\"onnen, darf die Auszahlungsmatrix von Spieler 1 nur positiv sein. Deshalb wird die Konstante \(c = {{ dict.added_value }}\) zur Matrix addiert, wodurch sich die Auszahlungsmatrix f\"ur Spieler 1 ergibt:
\begin{gather*}
\begin{pmatrix*}
 & {{col}} &\\
\end{pmatrix*}
\end{gather*}
Da die Voraussetzung einer nur positiven Auszahlungsmatrix von Spieler 1 bereits erf\"ullt ist, kann diese f\"ur die L\"osung durch den Simplex-Algorithmus herangezogen werden:
\begin{gather*}
\begin{pmatrix*}
 & {{col}} &\\
\end{pmatrix*}
\end{gather*}

Um die Simplex-Methode nutzen zu k\"onnen, wird folgendes Start-Tableau des dualen Problems aufgestellt:
\begin{equation}
\begin{array}{c}
{{ row }}\\
\end{array}
\begin{bmatrix}
\begin{array}{c|{{ dict.first_step.4 }}|c}
 {{ row }} &\\ \hline
 {{col}} &\\ \hline
 {{col}}& \\
\end{array}
\end{bmatrix}
\end{equation}
Hieraus ergeben sich folgende Simplex-Schritte mit den zugeh\"origen, ermittelten Pivots:

\begin{equation}
\begin{array}{c}
{{ row }}\\
\end{array}
\begin{bmatrix}
\begin{array}{c|{{ step.4 }}|c}
 {{ row }} &\\ \hline
 {{col}} &\\ \hline
 {{col}}& \\
\end{array}
\end{bmatrix}
\end{equation}
\begin{center}Pivot: {{ step.1 }}Kein weiteres Pivot-Element gefunden.\end{center}\\

Da der Spielwert nicht maximiert, sondern \(\frac{1}{G}\) minimiert wurde und die Konstante \(c = {{ dict.added_value }}\) zur urspr\"unglichen Auszahlungsmatrix addiert wurde, muss vom Ergebnis des Simplex-Verfahrens \(c\) wieder abgezogen werden und die inverse Transformation durchgef\"uhrt werden. Dadurch ergibt sich der Spielwert f\"ur Spieler 1:\\
\begin{center}{{ dict.game_value_simplex_player1 }}\end{center}\\
Um die Wahrscheinlichkeitsverteilung \"uber die reinen Strategien von Spieler 2 zu erhalten, m\"ussen die Werte aus der rechten Spalte mit dem inversen Zielwert aus der rechten Spalte und der untersten Zeile multipliziert werden. Dadurch ergibt sich die optimale Wahrscheinlichkeitsverteilung \"uber die reinen Strategien von Spieler 2 durch:\\
\begin{center}{{ dict.optimal_strategies_simplex_player2 }}\end{center}\\
Da ein Zwei-Personennullsummenspiel betrachtet wurde, ergibt sich der Spielwert f\"ur Spieler 2 zu:\\
\begin{center}{{ dict.game_value_simplex_player2 }}\end{center}\\
Die Dualit\"at des Problems, erlaubt es, die optimale Strategienkombination von Spieler 1 direkt aus der Zielfunktionszeile (letzte Zeile) abzulesen und ebenfalls mit dem inversen Ergebnis des Simplex-Verfahrens zu multiplizieren. Hieraus ergibt sich die Strategienkombination:\\
\begin{center}{{ dict.optimal_strategies_simplex_player1 }}\end{center}\\
Da der Spielwert nicht maximiert, sondern \(\frac{1}{G}\) minimiert wurde muss eine inverse Transformation durchgef\"uhrt werden. Dadurch ergibt sich der Spielwert f\"ur Spieler 1:\\
\begin{center}{{ dict.game_value_simplex_player1 }}\end{center}\\
Um die Wahrscheinlichkeitsverteilung \"uber die reinen Strategien von Spieler 2 zu erhalten, m\"ussen die Werte aus der rechten Spalte mit dem inversen Zielwert aus der rechten Spalte und der untersten Zeile multipliziert werden. Dadurch ergibt sich die optimale Wahrscheinlichkeitsverteilung \"uber die reinen Strategien von Spieler 2 durch:\\
\begin{center}{{ dict.optimal_strategies_simplex_player2 }}\end{center}\\
Da ein Zwei-Personennullsummenspiel betrachtet wurde, ergibt sich der Spielwert f\"ur Spieler 2 zu:\\
\begin{center}{{ dict.game_value_simplex_player2 }}\end{center}\\
Die Dualit\"at des Problems, erlaubt es, die optimale Strategienkombination von Spieler 1 direkt aus der Zielfunktionszeile (letzte Zeile) abzulesen und ebenfalls mit dem inversen Ergebnis des Simplex-Verfahrens zu multiplizieren. Hieraus ergibt sich die Strategienkombination:\\
\begin{center}{{ dict.optimal_strategies_simplex_player1 }}\end{center}\\
Au\"{ss}erdem kann das Spiel auf optimale Strategien gepr\"uft werden, indem ein lineares Gleichungssystem aufgestellt wird.\\
Da kein Nullsummenspiel vorliegt, k\"onnen die optimalen L\"osungen des Spiels nur \"uber die Betrachtung von Strategie-Kombinationen unter Ber\"ucksichtigungen der Bedingungen f\"ur ein Nash-Gleichgewicht ermittelt werden. Hierf\"ur m\"ussen zun\"achst die linearen Gleichungssysteme des Spiels aufgestellt werden.\\
Um den Spielwert von Spieler 1 und die optimale Strategien-Kombination von Spieler 2 zu ermitteln, wird folgendes Gleichungssystem gel\"ost:\\
\begin{gather*}
{{ row }}\\
\end{gather*}
Um den Spielwert von Spieler 2 und die optimale Strategien-Kombination von Spieler 1 zu ermitteln, wird folgendes Gleichungssystem gel\"ost:\\
\begin{gather*}
{{ row }}\\
\end{gather*}
Um nun alle optimalen L\"osungen des Spiels zu finden, m\"ussen alle Support-Mengen betrachtet werden. Nachfolgend werden alle Support-Mengen, die zu einem Nash-Gleichgewicht f\"uhren k\"onnen aufgef\"uhrt und analysiert. Strikt dominierte Strategien k\"onnen dabei keiner Support-Menge angeh\"oren, die zu einem Nash-Gleichgewicht f\"uhren soll, da sich der betroffene Spieler durch eine andere Strategie immer besser stellen kann.\\

Betrachten wir die Support-Menge von Spieler 1:\\
\begin{center}{{ ggw.0.0 }}\end{center}\\
und die Support-Menge von Spieler 2:\\
\begin{center}{{ ggw.0.1 }}\end{center}\\
So ergibt sich zur Bestimmung des Spielwerts von Spieler 1 und der optimalen Strategien-Kombination von Spieler 2 folgendes, zu l\"osendes, Gleichungssystem:\\
\begin{gather*}
{{ row }}\\
\end{gather*}
mit der L\"osung:\\
\begin{gather*}
{{ row }}\\
\end{gather*}
Hieraus ergibt sich auch der Spielwert von Spieler 1: \(w = {{ ggw.5 }}\) \\
Und f\"ur den Spielwert von Spieler 2 und die optimale Strategien-Kombination von Spieler 1 folgendes Gleichungssystem:\\
\begin{gather*}
{{ row }}\\
\end{gather*}
mit der L\"osung:\\
\begin{gather*}
{{ row }}\\
\end{gather*}
Hieraus ergibt sich auch der Spielwert von Spieler 1: \(w = {{ ggw.6 }}\) \\
Nun m\"ussen alle reinen Strategien von Spieler 1, die nicht in der gespielten Support-Menge liegen, betrachtet werden. Hierbei gilt es zu pr\"ufen, ob bei der errechneten Strategie-Kombination von Spieler 2, Spieler 1 die M\"oglichkeit h\"atte, sich eine h\"ohere Auszahlung zu sichern. Hierf\"ur m\"ussen die Auszahlungen aller nicht im Support liegenden reinen Strategien betrachtet und hinsichtlich Auszahlung \"uberpr\"uft werden:\\
\begin{gather*}
{{ eq }}\\Mit zugeh\"origer Auszahlung f\"ur Spieler 1: \(w = {{ notsupp.1 }}\)\\Welche geringer ist, als die, durch Spielen der Support-Menge, erzielte Auszahlung: \(w = {{ ggw.5 }}\)\\
\end{gather*}
Da f\"ur Spieler 1 alle reinen Strategien in der betrachteten Support-Menge liegen, sind keine weiteren Berechnungen n\"otig.\\
Nun m\"ussen alle reinen Strategien von Spieler 2, die nicht in der gespielten Support-Menge liegen, betrachtet werden. Hierbei gilt es zu pr\"ufen, ob bei der errechneten Strategie-Kombination von Spieler 1, Spieler 2 die M\"oglichkeit h\"atte, sich eine h\"ohere Auszahlung zu sichern. Hierf\"ur m\"ussen die Auszahlungen aller nicht im Support liegenden reinen Strategien betrachtet und hinsichtlich Auszahlung \"uberpr\"uft werden:\\
\begin{gather*}
{{ eq }}\\Mit zugeh\"origer Auszahlung f\"ur Spieler 2: \(w = {{ notsupp.1 }}\)\\Welche geringer ist, als die, durch Spielen der Support-Menge, erzielte Auszahlung: \(w = {{ ggw.6 }}\)\\
\end{gather*}
Da f\"ur Spieler 2 alle reinen Strategien in der betrachteten Support-Menge liegen, sind keine weiteren Berechnungen n\"otig.\\



\end{document}
